\documentclass[12pt,hyperref,]{ctexart}
\usepackage{lmodern}
\usepackage{amssymb,amsmath}
\usepackage{ifxetex,ifluatex}
\usepackage{fixltx2e} % provides \textsubscript
\ifnum 0\ifxetex 1\fi\ifluatex 1\fi=0 % if pdftex
  \usepackage[T1]{fontenc}
  \usepackage[utf8]{inputenc}
\else % if luatex or xelatex
  \ifxetex
    \usepackage{xltxtra,xunicode}
  \else
    \usepackage{fontspec}
  \fi
  \defaultfontfeatures{Mapping=tex-text,Scale=MatchLowercase}
  \newcommand{\euro}{€}
\fi
% use upquote if available, for straight quotes in verbatim environments
\IfFileExists{upquote.sty}{\usepackage{upquote}}{}
% use microtype if available
\IfFileExists{microtype.sty}{%
\usepackage{microtype}
\UseMicrotypeSet[protrusion]{basicmath} % disable protrusion for tt fonts
}{}
\usepackage[tmargin=1.8cm, bmargin=1.8cm, lmargin=2.1cm, rmargin=2.1cm]{geometry}
\ifxetex
  \usepackage[setpagesize=false, % page size defined by xetex
              unicode=false, % unicode breaks when used with xetex
              xetex]{hyperref}
\else
  \usepackage[unicode=true]{hyperref}
\fi
\usepackage[usenames,dvipsnames]{color}
\hypersetup{breaklinks=true,
            bookmarks=true,
            pdfauthor={},
            pdftitle={《R语言基础》教学大纲},
            colorlinks=true,
            citecolor=blue,
            urlcolor=blue,
            linkcolor=magenta,
            pdfborder={0 0 0}}
\urlstyle{same}  % don't use monospace font for urls
\setlength{\emergencystretch}{3em}  % prevent overfull lines
\providecommand{\tightlist}{%
  \setlength{\itemsep}{0pt}\setlength{\parskip}{0pt}}
\setcounter{secnumdepth}{5}

\title{《R语言基础》教学大纲}
\author{大数据分析基础系列}
\date{}
\usepackage{fontspec, xunicode, xltxtra}
\usepackage{xeCJK,ctex}
\usepackage{bm,xcolor}
\usepackage{fancyhdr}
\pagestyle{fancy}
\fancyhead[LE,LO]{《R语言基础》}
\fancyhead[RE,RO]{《大数据分析基础》系列}
\fancyfoot[RE,RO]{\thepage}
\fancyfoot[CO,CE]{上海数萃大数据学院}
\renewcommand{\headrule}{{\color{red}\hrule width\headwidth height\headrulewidth \vskip-\headrulewidth}}
\renewcommand{\footrule}{{\color{black}\vskip-\footruleskip\vskip-\footrulewidth \hrule width\headwidth height\footrulewidth\vskip\footruleskip}}
\renewcommand{\headrulewidth}{0.8pt}
\renewcommand{\footrulewidth}{0.6pt}
\renewcommand\contentsname{\bfseries 大纲目录}
\usepackage[sf,raggedright]{titlesec}
\usepackage{titletoc}
\titleformat{\section}{\bfseries\large\color{blue}}{\bfseries\thesection}{0.5em}{}
\titlespacing{\section}{0pt}{3.5ex plus .1ex minus .2ex}{1.5\wordsep}
\hypersetup{colorlinks=true, breaklinks=true, linkcolor=blue}

% Redefines (sub)paragraphs to behave more like sections
\ifx\paragraph\undefined\else
\let\oldparagraph\paragraph
\renewcommand{\paragraph}[1]{\oldparagraph{#1}\mbox{}}
\fi
\ifx\subparagraph\undefined\else
\let\oldsubparagraph\subparagraph
\renewcommand{\subparagraph}[1]{\oldsubparagraph{#1}\mbox{}}
\fi

\begin{document}
\maketitle

{
\setcounter{tocdepth}{2}
\tableofcontents
}
\newpage

\noindent\textbf{课程名称}: R语言基础

\noindent\textbf{适合专业}:
数据科学与大数据技术、大数据技术与应用、统计学、应用统计学、统计统计、应用数学等本科专业

\section{课程目的、任务}

本课程是为数据科学与大数据技术及相关专业学生开设的一门课程,《R语言基础》是一门基于R进行数据分析的基础课程,本课程的目的是帮助学生从零开始对R语言有全方位了解,掌握R语言数据处理、统计分析和可视化最为基础知识、技术和基本应用,为使用R语言进行深入的数据分析与行业应用打下扎实的基础。本课程是后期统计分析、机器学习、可视化等相关课程的预修课程。

\section{课程简介}

\begin{itemize}
\item
  R语言是一个自由、免费、源代码开放的语言环境。它功能完备,语言灵活且不依赖操作系统,是一个用于统计计算和统计制图的优秀工具。在2010
  年,美国统计协会( ASA)授予R
  语言为(第⼀届)统计计算及图形奖。截止目前为止,R是数据科学领域使用用户最多的数据挖掘与编程语言,且其用户数量在不断增加。R已经成为几乎所有国内外高校统计类课程标配的教学软件,R的书集遍及经济、金融、生物、医学、生态、电子商务、航空、旅游、心理、法律等各个领域,并成大数据分析的宠儿:
  覆盖机器学习、人工智能和可视化等。R语言在国内业界也逐步成为主流的分析工具,每年全国性的R语言会议推进R的迅猛发展。
\item
  本课程共5大模块,从R入门、R数据集创建与管理、R绘图初步、R数据探索与比较分析和R统计建模等方面,让零基础学员从各个角度对R的使用进行学习。
\item
  本课程为基础课程,学员可以通过学习对R语言进行全方位的了解与掌握,并为R的进阶学习与实际数据分析做好基础铺垫。
\end{itemize}

\section{教学方式和实践环节的特色}

\begin{itemize}
\item
  教学方式:课堂讲授,配合大数据平台演示。
\item
  实践环节:

  \begin{enumerate}
  \def\labelenumi{\arabic{enumi}.}
  \item
    课堂讲授中提供一定的思考题供学生练习或讨论;
  \item
    课后布置作业,作业量2---3小时,作业中以基本练习为主。
  \end{enumerate}
\end{itemize}

\section{教材及参考书目}

\begin{itemize}
\item
  建议教材:

  \begin{enumerate}
  \def\labelenumi{\arabic{enumi}.}
  \item
    R.I. Kabacoff著, 高涛,肖南, 陈钢译, R语言实战(R in Action: Data
    Analysis and Graphics with R), 人民邮电出版社, 2013.
  \item
    徐珉久{[}韩{]}, 武传海 (译). R语言与数据分析实战.
    中国工信出版集团,人民邮电出版社, 2017.1
  \end{enumerate}
\item
  参考书目:

  \begin{enumerate}
  \def\labelenumi{\arabic{enumi}.}
  \item
    汤银才, R语言与统计分析, 高等教育出版社. 2008
  \item
    Joseph Adler, R in a Nutshell, California: O'Reilly Meda, Inc., 2nd
    Ed. 2012.
  \item
    Jared P. Lander, R for Everyone: Advanced Analytics and Graphics.
    New York: Addison-Wesley. 2014.
  \item
    Hadley Wickham, Garrett Grolemund, R for Data Science -
    Import,Tidy,Transform,Visualize, and Model data, O'Reilly. 2017.
  \item
    Jim Albert, M. Rizzo, R by example, Springer, 2012.
  \item
    Prabhanjan N. Tattar, Suresh Ramaiah, B.G. Manjunath. A Course in
    Statistics with R(ACSWR), for master students. Wiley. 2016. {[}*{]}
  \item
    Norman Matloff. The art of R programming, No Starch Press, inc.,
    2011.
  \item
    Paul Murrell, R Graphics, Chapman \& Hall/CRC. 2006.
  \item
    Deepayan Sarkar, Lattice: Multivariate Data Visualization with R,
    Springer. 2008.
  \item
    Hadley Wickham, ggplot2: Elegant Graphics for Data Analysis,
    Springer. 2009. (中译本: 殷腾飞,统计之都, 2014)
  \item
    Donato Teutonico, ggplot2 Essentials, PACKT, 2015.
  \item
    Alboukadel Kassambara, ggplot2 - Guide to Create Beautiful Graphics
    in R (2nd Ed). 2013.
  \item
    Mark P.J. van der Loo, Edwin de Jonge, Learning Rstudio for R
    Statistical Computing. PACKT, 2012.
  \item
    John Verzani, Getting Started with Rstudio, OReilly, 2011.
  \item
    Nicholas J. Horton and Ken Kleinman. Using R and RStudio for Data
    Management Statistical Analysis and Graphics(2nd Ed.),2015.
  \end{enumerate}
\end{itemize}

\section{考核方式与评价结构比例}

\begin{itemize}
\item
  考核内容为:平时表现、作业完成情况、平时测验、期终闭卷考试。
\item
  评价结构比例:

  \begin{enumerate}
  \def\labelenumi{\arabic{enumi}.}
  \tightlist
  \item
    总评成绩由平时成绩和考试成绩两部分组成,一般平时成绩占40\%,考试成绩占60\%。
  \end{enumerate}

  2.平时成绩由平时表现、作业、平时测验成绩组成,一般平时表现占10\%,作业占30\%,平时测验成绩占60\%。
\end{itemize}

\section{讲授大纲}

\textbf{M1:R入门}

\begin{enumerate}
\def\labelenumi{\arabic{enumi}.}
\item
  大数据与数据科学

  \begin{itemize}
  \tightlist
  \item
    数据科学与分析工具
  \item
    R及其优势
  \item
    R安装与配置
  \item
    R包安装与使用
  \item
    R资源与帮助
  \end{itemize}
\item
  R快速入门教程

  \begin{itemize}
  \tightlist
  \item
    R中的基本语法
  \item
    R中的数据对象及其属性
  \item
    R的工作空间与管理
  \item
    R编程基础
  \item
    R程序调试
  \end{itemize}
\item
  R编辑器与RStudio

  \begin{itemize}
  \tightlist
  \item
    R常用编辑器
  \item
    Rstudio功能与使用技巧
  \item
    Rstudio进阶
  \item
    项目管理
  \end{itemize}
\item
  Rmarkdown介绍

  \begin{itemize}
  \tightlist
  \item
    Rmarkdown
  \item
    文学化统计编程
  \end{itemize}
\end{enumerate}

\textbf{M2:R数据集创建与管理}

\begin{enumerate}
\def\labelenumi{\arabic{enumi}.}
\item
  数据集的创建与处理

  \begin{itemize}
  \item
    常用数据对象与创建
  \item
    数据的读取
  \item
    数据的存储
  \item
    数据集的处理
  \item
    数据集的合并与子集提取
  \item
    缺失值的处理
  \item
    数据表数据的切片、切块与组合
  \end{itemize}
\item
  向量的操作

  \begin{itemize}
  \item
    合并数据框的行(向量)与列(向量)
  \item
    apply系列函数
  \item
    数据分组并调用函数
  \item
    数据拆分与合并
  \item
    数据排序
  \item
    访问数据中的列
  \item
    查找符合条件的数据索引
  \item
    分组运算
  \end{itemize}
\item
  数据对象的其他操作

  \begin{itemize}
  \item
    赋值与常用运算
  \item
    基本的数学运算
  \item
    用于矩阵的运算
  \item
    正则表达式
  \item
    与统计分布相关的函数
  \end{itemize}
\end{enumerate}

\textbf{M3:R绘图初步}

\begin{enumerate}
\def\labelenumi{\arabic{enumi}.}
\item
  基本的绘图命令

  \begin{itemize}
  \tightlist
  \item
    大趋势:信息可视化
  \item
    R绘图基础: 低级与高级绘图命令
  \item
    基本绘图函数: plot, points, lines, curve,par
  \item
    绘图三要素设置详解(颜色,点型,线型)
  \item
    绘图信息补充(title,text,legend,axis)
  \item
    R窗口和图形设备(X11,pdf,png)
  \end{itemize}
\item
  一维数据的可视化

  \begin{itemize}
  \tightlist
  \item
    常用统计分布与4类函数
  \item
    一维离散变量的分布图示
  \item
    一维连续变量的分布图示
  \item
    一维连续分布诊断图
  \item
    非参数密度估计与展示
  \end{itemize}
\item
  多维数据的可视化

  \begin{itemize}
  \tightlist
  \item
    二个离散变量的分布图示
  \item
    二个混合变量的分布图示
  \item
    二个连续变量的分布图示
  \item
    多变量的可视化
  \end{itemize}
\end{enumerate}

\textbf{M4:R统计分析初步}

\begin{enumerate}
\def\labelenumi{\arabic{enumi}.}
\item
  数据分析基础
\item
  R中常用的统计函数
\item
  描述性统计分析

  \begin{itemize}
  \tightlist
  \item
    常用描述性统计量及其计
  \item
    单个连续型变量描述性统计量的获取
  \item
    分组计算描述性统计量
  \end{itemize}
\item
  相关性分析
\item
  相关性检验
\end{enumerate}

\textbf{M5:R统计建模}

\begin{enumerate}
\def\labelenumi{\arabic{enumi}.}
\item
  回归模型

  \begin{itemize}
  \tightlist
  \item
    lm()函数中的公式表示
  \item
    一元线性回归
  \item
    多元线性回归
  \item
    回归预测
  \end{itemize}
\item
  分类数据的统计推断

  \begin{itemize}
  \tightlist
  \item
    广义线性模型概述
  \item
    glm()函数介绍
  \item
    Logistic回归
  \item
    Poisson回归
  \end{itemize}
\item
  模型的检验与比较

  \begin{itemize}
  \tightlist
  \item
    回归模型诊断
  \item
    变量的选择
  \item
    模型比较
  \item
    异常值判断
  \item
    预测与交叉验证
  \end{itemize}
\end{enumerate}

\end{document}
